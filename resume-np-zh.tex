%%
%% Copyright (c) 2018-2019 Weitian LI <wt@liwt.net>
%% CC BY 4.0 License
%%
%% Created: 2018-04-11
%%

% Chinese version
\documentclass[zh]{resume}

% Adjust icon size (default: same size as the text)
\iconsize{\Large}

% % File information shown at the footer of the last page
% \fileinfo{%
%   % \faCopyright{} 2018--2020, Weitian LI \hspace{0.5em}
%   % \creativecommons{by}{4.0} \hspace{0.5em}
%   % \githublink{liweitianux}{resume} \hspace{0.5em}
%   \faEdit{} \today
% }

\name{鹏}{倪}

\keywords{Bioinformatics, Epigenetics, Python,R}

% \tagline{\icon{\faBinoculars}} <position-to-look-for>}
% \tagline{<current-position>}

\profile{
  \mobile{152-7319-1121}
  \email{nipeng@csu.edu.cn}
  \github{PengNi} \\
  \university{中南大学 \textbullet 计算机学院}
  \degree{计算机科学与技术 \textbullet 博士研究生}
  \birthday{1991-12-26}
  \address{湖南 \textbullet 长沙}
  % Custom information:
  % \icontext{<icon>}{<text>}
  % \iconlink{<icon>}{<link>}{<text>}
}

\photo{9em}{nipeng_csu.jpg}

\begin{document}
\makeheader

%======================================================================
% Summary & Objectives
%======================================================================
{\onehalfspacing\hspace{2em}%
计算机科学与技术专业、生物信息学方向博士研究生,导师王建新教授。目前研究方向为利用长读数测序技术进行碱基修饰检测,分别基于纳米孔和PacBio长读数测序技术,开发了\link{https://github.com/bioinfomaticsCSU/deepsignal}{DeepSignal}和\link{https://github.com/PengNi/deepsignal-plant}{DeepSignal-plant}、\link{https://github.com/PengNi/ccsmeth}{ccsmeth}等检测DNA 5mC甲基化修饰的计算工具,相关文章发表在生物信息学领域权威期刊。
\par}

%======================================================================
\sectionTitle{技能和语言}{\faWrench}
%======================================================================
\begin{competences}
\comptence{编程}{%
    Python, R/ggplot2, Java, Nextflow, Rust (学习中)
  }
  \comptence{工具}{%
    SSH, Git, Bash/Shell
  }
  \comptence{\icon{\faLanguage} 语言}{
    \textbf{英语} --- 读写(熟练),听说(日常交流)
  }
\end{competences}

%======================================================================
\sectionTitle{教育背景}{\faGraduationCap}
%======================================================================
\begin{educations}
  \education%
    {2017.09}%
    [至今]%
    {中南大学}%
    {计算机学院}%
    {计算机科学与技术}%
    {博士研究生(预计毕业时间:2022.12)}

  \separator{0.5ex}
  \education%
    {2018.11}%
    [2019.04]%
    {克莱姆森大学}%
    {计算机学院}%
    {罗峰老师团队}%
    {访问学者}

  \separator{0.5ex}
  \education%
    {2014.09}%
    [2017.06]%
    {中南大学}%
    {信息科学与工程学院}%
    {计算机科学与技术}%
    {硕士}

  \separator{0.5ex}
  \education%
    {2010.09}%
    [2014.06]%
    {山东大学}%
    {计算机科学与技术学院}%
    {计算机科学与技术}%
    {本科}
\end{educations}

% %======================================================================
% \sectionTitle{计算机技能}{\faCogs}
% %======================================================================
% \begin{itemize}
%   \item xxx
% \end{itemize}

% %======================================================================
% \sectionTitle{个人项目}{\faCode}
% %======================================================================

%======================================================================
\sectionTitle{科研成果}{\faAtom}
%======================================================================
\begin{itemize}
  \item 参与研究课题
  \begin{itemize}
    \item[>] 复杂生物医学数据处理方法及应用研究(联合基金重点项目)
    \item[>] 单倍型群体基因组组装及其疾病关联分析方法研究(专项项目)
  \end{itemize}
  \item 开发方法工具
  \begin{itemize}
    \item[>] 开发基于纳米孔测序数据的DNA 5mC检测方法\link{https://github.com/bioinfomaticsCSU/deepsignal}{DeepSignal}和\link{https://github.com/PengNi/deepsignal-plant}{DeepSignal-plant}
    \item[>] 开发基于PacBio测序数据的DNA 5mC检测方法\link{https://github.com/PengNi/ccsmeth}{ccsmeth}
    \item[>] 开发基于PacBio测序数据和Spark并行技术的DNA甲基化检测流程\link{https://github.com/PengNi/basemods_spark}{basemods\_spark}
  \end{itemize}
  \item 发表第一作者论文4篇
\end{itemize}

%======================================================================
% Papers / Publications
\sectionTitle{发表论文}{\faFileAlt}
%======================================================================
\begin{itemize}
  \item \textbf{第一作者论文}
  \begin{enumerate}
    %\small
    \item \textbf{Peng Ni}, Jinrui Xu, Zeyu Zhong, Jun Zhang, Neng Huang, Fan Nie, Feng Luo, and Jianxin Wang. \enquote{DNA 5-methylcytosine detection and methylation phasing using PacBio circular consensus sequencing.} {\it bioRxiv} (2022). (preprint)
    \item \textbf{Peng Ni}, Neng Huang, Fan Nie, Jun Zhang, Zhi Zhang, Bo Wu, Lu Bai, Wende Liu, Chuan-Le Xiao, Feng Luo, and Jianxin Wang. \enquote{Genome-wide detection of cytosine methylations in plant from Nanopore data using deep learning.} {\it Nature Communications} 12, no. 1 (2021): 1-11. (SCI; JCR 1区; IF=14.919)
    \bigskip
    \medskip
    \item \textbf{Peng Ni}, Jianxin Wang, Ping Zhong, Yaohang Li, Fang-Xiang Wu, and Yi Pan. \enquote{Constructing Disease Similarity Networks Based on Disease Module Theory.} {\it IEEE/ACM Transactions on Computational Biology and Bioinformatics} 17, no. 03 (2020): 906-915. (SCI; JCR 1区; IF=3.710)
    \item \textbf{Peng Ni}*, Neng Huang*, Zhi Zhang, De-Peng Wang, Fan Liang, Yu Miao, Chuan-Le Xiao, Feng Luo, and Jianxin Wang. \enquote{DeepSignal: detecting DNA methylation state from Nanopore sequencing reads using deep-learning.} {\it Bioinformatics} 35, no. 22 (2019): 4586-4595. (共同第一作者; SCI; JCR 1区; IF=6.937)
  \end{enumerate}
  \medskip
  \item \textbf{其它论文}
  \begin{enumerate}
    %\small
    \item Neng Huang, Fan Nie, \textbf{Peng Ni}, Xin Gao, Feng Luo, and Jianxin Wang. \enquote{BlockPolish: accurate polishing of long-read assembly via block divide-and-conquer.} {\it Briefings in bioinformatics} 23, no. 1 (2022): bbab405. (SCI; JCR 1区; IF=11.622)
    \item Neng Huang, Fan Nie, \textbf{Peng Ni}, Feng Luo, Xin Gao, and Jianxin Wang. \enquote{NeuralPolish: a novel Nanopore polishing method based on alignment matrix construction and orthogonal Bi-GRU Networks.} {\it Bioinformatics} 37, no. 19 (2021): 3120-3127. (SCI; JCR 1区; IF=6.937)
    \item Haochen Zhao, Guihua Duan, \textbf{Peng Ni}, Cheng Yan, Yaohang Li, and Jianxin Wang. \enquote{RNPredATC: a deep residual learning-based model with applications to the prediction of drug-ATC code association.} {\it IEEE/ACM Transactions on Computational Biology and Bioinformatics} (2021). (SCI; JCR 1区; IF=3.710)
    \item Neng Huang, Fan Nie, \textbf{Peng Ni}, Feng Luo, and Jianxin Wang. \enquote{Sacall: a neural network basecaller for oxford nanopore sequencing data based on self-attention mechanism.} {\it IEEE/ACM Transactions on Computational Biology and Bioinformatics} (2020). (SCI; JCR 1区; IF=3.710)
    \item Huimin Luo, Jianxin Wang, Min Li, Junwei Luo, \textbf{Peng Ni}, Kaijie Zhao, Fang-Xiang Wu, and Yi Pan. \enquote{Computational Drug Repositioning with Random Walk on a Heterogeneous Network.} {\it IEEE/ACM Transactions on Computational Biology and Bioinformatics} 16, no. 06 (2019): 1890-1900. (SCI; JCR 1区; IF=3.710)
    \item Min Li, \textbf{Peng Ni}, Xiaopei Chen, Jianxin Wang, Fang-Xiang Wu, and Yi Pan. \enquote{Construction of Refined Protein Interaction Network for Predicting Essential Proteins.} {\it IEEE/ACM Transactions on Computational Biology and Bioinformatics} 16, no. 04 (2019): 1386-1397. (SCI; JCR 1区; IF=3.710)
    \item Cheng Yan, Jianxin Wang, \textbf{Peng Ni}, Wei Lan, Fang-Xiang Wu, and Yi Pan. \enquote{DNRLMF-MDA: Predicting microRNA-Disease Associations Based on Similarities of microRNAs and Diseases.} {\it IEEE/ACM Transactions on Computational Biology and Bioinformatics} 16, no. 01 (2019): 233-243. (SCI; JCR 1区; IF=3.710)
  \end{enumerate}
\end{itemize}

%======================================================================
% Awards / Scholarships / Certificates
\sectionTitle{获奖及证书}{\faAward}
%======================================================================
\begin{entries}
  \entry{2018.11}%
    {中南大学博士研究生国家奖学金}
\end{entries}

% %======================================================================
% \sectionTitle{实习经历}{\faBriefcase}
% %======================================================================
% \begin{experiences}
%   \experience%
%     [20xx.xx]%
%     {20xx.xx}%
%     {xxx}%
%     [\begin{itemize}
%       \item xxx
%     \end{itemize}]

%   \separator{0.5ex}
% \end{experiences}

\end{document}
